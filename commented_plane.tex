%3D coordinate system with plane
\documentclass[tikz,border=5]{standalone}

%Drawing
\usepackage{tikz}

%Tikz Library
\usetikzlibrary{calc}

%Notation
\usepackage{physics}

\begin{document}

% All numbers must be in [0,1] to fall in the plane drawn
\def\t{0.9}
\def\s{0.04}
\def\tt{0.3}
\def\ss{0.7}
\def\ttt{0.5}
\def\sss{0.5}

\begin{tikzpicture}[x={(1cm,0.4cm)}, y={(8mm, -3mm)}, z={(0cm,1cm)}, line cap=round, line join=round]
%Coordinates
%Plane Vertex Points
\coordinate (x1) at (-2,2,3);
\coordinate (x2) at (2,2,5);
\coordinate (x3) at (2,-2,5);
\coordinate (x4) at (-2,-2,3);
%Vectors Parallel to Plane
\coordinate (n1) at ($(x2) - (x1)$);
\coordinate (n2) at ($(x2) - (x3)$);
%Points on Plane
\coordinate (x5) at ($(x1) + \s*(n1) - \t*(n2)$);
\node[outer sep = 1pt, inner sep = 1pt] (x6) at ($(x1) + \ss*(n1) - \tt*(n2)$) {};
\coordinate (x7) at ($(x1) + \sss*(n1) - \ttt*(n2)$);
%Beginning of Axis
\coordinate (O) at (0,0,0);
%Random Point
\node[outer sep = 1pt, inner sep = 1pt] (P) at (2.5,1,5.5) {};

%Axis
%Black 'O' origin label, Origin point
\draw[-latex] (-2.5,0,0) -- (2.5,0,0) node[pos = 1.05] {$x$};
\draw[-latex] (0,-3.5,0) -- (0,3.5,0) node[pos = 1.05] {$y$};
\draw[-latex] (0,0,0) -- (0,0,7) node[pos = 1.05] {$z$};
%Black 'O' origin label, Origin point
\draw[draw=black, fill=black] (O) circle (1pt) node[below] {${O}$};

%Point on Plane
%Gray plane, Gray line on plane
\draw[-latex, thick] (O) -- (x6) node[pos=0.45, shift={(0.1,0.3)}] {$\vb{r_o}$};
%Plane
\path[draw=black, fill=black!20, thick, opacity = 0.8] (x1) -- (x2) -- (x3) -- (x4) -- (x1);
\node[shift={(-0.45,0.6)}] at (x3) {$\Pi$};
%Perpendicular Vector
\draw[-latex, thick] (x5) -- ($(x5)!0.07!(-8,0,24)$) node[pos=0.5, shift={(-0.2,-0.1)}] {$\vb{u}$};
%Point on Plane
%Gray line on plane, Gray line from origin to plane
\draw[draw=black, fill=black] (x6) circle (1pt) node[above right] {${P_o}$};

%Z-axis Section
%Gray plane, Gray line on plane
\draw[draw=black, fill=black] (x7) circle (0.5pt);
%Gray line on plane, Gray line from origin to plane
\draw (x7) -- (0,0,6.5);

\end{tikzpicture}

\end{document}